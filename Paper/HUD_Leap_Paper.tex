\documentclass{sigchi-ext}
% Please be sure that you have the dependencies (i.e., additional
% LaTeX packages) to compile this example.
\usepackage[T1]{fontenc}
\usepackage{textcomp}
\usepackage[scaled=.92]{helvet} % for proper fonts
\usepackage{graphicx} % for EPS use the graphics package instead
\usepackage{balance}  % for useful for balancing the last columns
\usepackage{booktabs} % for pretty table rules
\usepackage{ccicons}  % for Creative Commons citation icons
\usepackage{ragged2e} % for tighter hyphenation

% Some optional stuff you might like/need.
% \usepackage{marginnote} 
% \usepackage[shortlabels]{enumitem}
% \usepackage{paralist}
% \usepackage[utf8]{inputenc} % for a UTF8 editor only

%% EXAMPLE BEGIN -- HOW TO OVERRIDE THE DEFAULT COPYRIGHT STRIP --
% \copyrightinfo{Permission to make digital or hard copies of all or
% part of this work for personal or classroom use is granted without
% fee provided that copies are not made or distributed for profit or
% commercial advantage and that copies bear this notice and the full
% citation on the first page. Copyrights for components of this work
% owned by others than ACM must be honored. Abstracting with credit is
% permitted. To copy otherwise, or republish, to post on servers or to
% redistribute to lists, requires prior specific permission and/or a
% fee. Request permissions from permissions@acm.org.\\
% {\emph{CHI'14}}, April 26--May 1, 2014, Toronto, Canada. \\
% Copyright \copyright~2014 ACM ISBN/14/04...\$15.00. \\
% DOI string from ACM form confirmation}
%% EXAMPLE END

% Paper metadata (use plain text, for PDF inclusion and later
% re-using, if desired).  Use \emtpyauthor when submitting for review
% so you remain anonymous.
\def\plaintitle{SIGCHI Extended Abstracts Sample File: Note Initial
  Caps} \def\plainauthor{First Author, Second Author, Third Author,
  Fourth Author, Fifth Author, Sixth Author}
\def\emptyauthor{}
\def\plainkeywords{Authors' choice; of terms; separated; by
  semicolons; include commas, within terms only; required.}
\def\plaingeneralterms{Documentation, Standardization}

\title{Prototyping a gesture controlled \underline{H}ead \underline{U}p \underline{D}isplay using a Leap Motion}

\numberofauthors{2}
% Notice how author names are alternately typesetted to appear ordered
% in 2-column format; i.e., the first 4 autors on the first column and
% the other 4 auhors on the second column. Actually, it's up to you to
% strictly adhere to this author notation.
\author{%
  \alignauthor{%
    \textbf{Daniel Brand}\\
    \affaddr{University of Salzburg} \\
    \affaddr{Salzburg, 5020, AUT} \\
    \email{Daniel.Brand@stud.sbg.ac.at} 
}
\alignauthor{%
    \textbf{Kevin B\"uchele}\\
    \affaddr{University of Salzburg}\\
    \affaddr{Salzburg, 5020, AUT}\\
    \email{Kevin.Buechele@stud.sbg.ac.at} 
} \vfil 
}

% Make sure hyperref comes last of your loaded packages, to give it a
% fighting chance of not being over-written, since its job is to
% redefine many LaTeX commands.
\definecolor{linkColor}{RGB}{6,125,233}
\hypersetup{%
  pdftitle={\plaintitle},
%  pdfauthor={\plainauthor},
  pdfauthor={\emptyauthor},
  pdfkeywords={\plainkeywords},
  bookmarksnumbered,
  pdfstartview={FitH},
  colorlinks,
  citecolor=black,
  filecolor=black,
  linkcolor=black,
  urlcolor=linkColor,
  breaklinks=true,
}

% \reversemarginpar%

\begin{document}

\maketitle

% Uncomment to disable hyphenation (not recommended)
% https://twitter.com/anjirokhan/status/546046683331973120
\RaggedRight{} 

% Do not change the page size or page settings.
\begin{abstract}
In todays cars, the design of a Head Up Display (HUD) and
the arrangement of its components are rather fixed. This paper provides an approach
how a Leap Motion can be used to achieve a gesture controled HUD. Furthermore, some aspects that need to be considered
when implementing a modifyable HUD for cars are discussed.
\end{abstract}

\keywords{LeapMotion; HUD; Head Up Display; Automotive; Gesture recognition}

\section{Introduction}
During the last years, the HUD display has been established in middle and high class cars.
However, Head Up Displays are rather static, in particular, the position of each element in the display
is fixed.\linebreak

Nowadays, users of any devices appreciate an interface which can be modifyed.
Since buttons or similar controls may distract the driver from looking at the road,
these types of controling the interface are rather bad choices.\linebreak

After doing some research, we came to the conclusion that a touchless control
(similar to other approaches to control a infotainment system)
may be a reasonable goal we want to achieve.\linebreak

A classic device to implement touchless controls is the Leap Motion. We use the Leap Motion
and furthermore gestures which are easy for the user and also easy to be tracked by the device.
In this paper, we will discuss problems and issue we have faced regarding the position of
the Leap Motion, which gestures to use and more important which settings/requirements have to be fixed
in order to get the system working smoothly. By the end, we can select items within a HUD
without any other devices such as a touchpad, etc.

\section{The system's environment}
The Leap Motion is a classic sensor device to track the hands of a user and therefore his gestures.
The detection of the hand works with an infrared camera which has the advantage that environment light
which would generate noise in the information are reduced. The sensor recognizes each finger including
bones and joints, which gives the possibility to process data of a single finger, which we are using in
our approach (details are described later). So far, our system has been configured to work with a right hand only.
For now, there is no support for left handers.\linebreak

Our HUD will be simulated with the help of the JavaFX framework. In order to test which settings fit the most for this use,
we are able to dynamically display 3-5 elements in our HUD, simply represented with numbers between 1 and 3 (respectively 5).
Certain gestures above the Leap Motion will cause an interaction with the JavaFX window.
The Leap Motion is set up to track 2 major gestures: pointing at a certain field (see figure)
and selecting it by moving the pointing finger slightly in direction to the HUD. In the case of our prototype,
the Leap Motion Listener and the display are running on the same computer.\linebreak

The angles between finger, Leap Motion and HUD have to be ajusted accordingly. This is because we are using the
interface with the right hand, more specificly the right pointing finger, only. Obviously, the anatomy of our hand
allows a slightly bigger movement range of the finger to the left instead of the right. This needed some testing, and we came to
the conclusion that the angles have to be ajusted within the program in order to achieve a satisfying control of the HUD.

\section{Test setup}
Each test person needs to complete two test sessions. The difference between this sessions are the amount of elements in our HUD. After some pre-tests, we now only include test sessions with three and four elements. Using five elements did not lead to pleasant results since selecting one out of five elements was way too difficult.  \linebreak

Each session consists of 20 steps. In each step, the program marks a random element by coloring the number purble. The user should now try and select the marked element by moving the pointing finger accordingly (left or right). If the desired element is marked, the user can then select it by moving the pointing finger slightly forward. After successfully (or not) selecting an element, the test sequence is continued with the next element. \linebreak

The result of each step is recorded and (after completing the sequence) summarized in a log file. The logfile provides detailed test results in order to validate our prototype. The first few lines give a summary about basic settings and results such as implemented angles, test sequence, touched elements in the test sequence and the duration of the whole test. The second bigger a part of the log file gives an exact result about which angle was hit after each step (with respect of the angle window of the element).


\section{Test results}
tbd

\bibliographystyle{SIGCHI-Reference-Format}
\bibliography{sample}

\end{document}

%%% Local Variables:
%%% mode: latex
%%% TeX-master: t
%%% End:
